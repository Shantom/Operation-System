\documentclass{article}

\usepackage{xeCJK}
\usepackage{geometry}
\usepackage{listings}
\usepackage{graphicx}
\usepackage{hyperref}
\geometry{a4paper,centering,scale=0.8}

\begin{document}
\title{Reader Writer Problem}
\author{史柠玮-2014211283}
\date{\today}
\maketitle
\tableofcontents
\pagebreak

\section{实验目的}
深刻理解读者写者问题

\section{实验要求}

\subsection{基本要求}

创建一个包含n 个线程的控制进程。用这n 个线程来表示n
个读者或写者。每个线程按相应测试数据文件的要求,进行读写操作。请用信号量
机制分别实现读者优先和写者优先的读者-写者问题。
读者-写者问题的读写操作限制:
\begin{enumerate}
\item 写-写互斥;
\item 读-写互斥;
\item 读-读允许;
\end{enumerate}


\subsection{附加限制}
\begin{itemize}
\item 读者优先的附加限制:如果一个读者申请进行读操作时已有另一读者正在进行读操作,
则该读者可直接开始读操作。
\item 写者优先的附加限制:如果一个读者申请进行读操作时已有另一写者在等待访问共享资
源,则该读者必须等到没有写者处于等待状态后才能开始读操作。
\end{itemize}



\subsection{运行结果显示要求}

要求在每个线程创建、发出读写操作申请、开始读写操作和结束读
写操作时分别显示一行提示信息,以确信所有处理都遵守相应的读写操作限制。

\subsection{测试数据文件格式}

测试数据文件包括n 行测试数据,分别描述创建的n 个线程是读者还是写者,以及读写
操作的开始时间和持续时间。每行测试数据包括四个字段,各字段间用空格分隔。

\begin{enumerate}
\item 第一字段为一个正整数,表示线程序号.
\item 第二字段表示相应线程角色,R 表示读者是,W 表示写者。
\item 第三字段为一个正数,表示读写操作的开始时间。线程创建后,延时相应时间(单位为秒)
后发出对共享资源的读写申请。
\item 第四字段为一个正数,表示读写操作的持续时间。当线程读写申请成功后,
  开始对共享资源的读写操作,该操作持续相应时间后结束,并释放共享资源.
\end{enumerate}

下面是一个测试数据文件的例子:
\begin{center}
  1 R 3 5\\
2 W 4 5\\
3 R 5 2\\
4 R 6 5\\
5 W 5.1 3
\end{center}

\section{实验环境}
\begin{itemize}
\item Fedora 25 64位
\item Pycharm (IDE)
\item Python 3.5
\end{itemize}


\section{实验结果}

注\footnote{代码见附件}

\begin{itemize}
\item 读者优先:\\
  \includegraphics[width=15cm]{1.png}
\item 写者优先:\\
  \includegraphics[width=15cm]{2.png}
\end{itemize}

\end{document}
